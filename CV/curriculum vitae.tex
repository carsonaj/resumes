\documentclass{resume} % Use the custom resume.cls style

\usepackage{hyperref}
\usepackage{xcolor}
\usepackage{amsmath,amsthm,amssymb,amsfonts,setspace}
\usepackage[shortlabels]{enumitem}

\newcommand{\al}{\alpha}
\newcommand{\be}{\beta} 
\newcommand{\del}{\delta} 
\newcommand{\Del}{\Delta}
\newcommand{\lam}{\lambda}  
\newcommand{\Lam}{\Lambda} 
\newcommand{\ep}{\epsilon}
\newcommand{\sig}{\sigma} 
\newcommand{\om}{\omega}
\newcommand{\Om}{\Omega}
\newcommand{\C}{\mathbb{C}}
\newcommand{\N}{\mathbb{N}}
\newcommand{\E}{\mathbb{E}}
\newcommand{\Z}{\mathbb{Z}}
\newcommand{\R}{\mathbb{R}}
\newcommand{\Q}{\mathbb{Q}}
\renewcommand{\P}{\mathbb{P}}
\newcommand{\MA}{\mathcal{A}}
\newcommand{\MB}{\mathcal{B}}
\newcommand{\MF}{\mathcal{F}}
\newcommand{\MG}{\mathcal{G}}
\newcommand{\MJ}{\mathcal{J}}
\newcommand{\ML}{\mathcal{L}}
\newcommand{\MN}{\mathcal{N}}
\newcommand{\MS}{\mathcal{S}}
\newcommand{\MP}{\mathcal{P}}
\newcommand{\ME}{\mathcal{E}}
\newcommand{\MT}{\mathcal{T}}
\newcommand{\MM}{\mathcal{M}}

\usepackage[left=0.75in,top=0.6in,right=0.75in,bottom=0.6in]{geometry} % Document margins
\newcommand{\tab}[1]{\hspace{.2667\textwidth}\rlap{#1}}
\newcommand{\itab}[1]{\hspace{0em}\rlap{#1}}
\name{Carson James} % Your name
\address{(+1)~405~315~5881 \\ carson.a.james@gmail.com} % Your phone number and email
\address{201 Darwin Rd \\ Edmond, Oklahoma 73034} % Your address
%\address{123 Pleasant Lane \\ City, State 12345} % Your secondary addess (optional)


\begin{document}

%----------------------------------------------------------------------------------------
%	Education
%----------------------------------------------------------------------------------------
\begin{rSection}{Education}
Phd Statistics \hfill {\em August 2020 - May 2025} \\
{\bf Texas A\&M University}  \hfill {GPA: 4.0} 

MSc Mathematics \hfill {\em August 2016 - May 2018} \\
{\bf Oklahoma State University}  \hfill {GPA: 4.0} 

BA Mathematics \hfill {\em August 2013 - May 2016} \\
{\bf Oklahoma State University} \hfill {GPA: 3.929} \\
\end{rSection}


%----------------------------------------------------------------------------------------
%	Relevant Courses
%------------------------------------------------------------
\begin{rSection}{Relevant Courses and Material Covered}
\begin{itemize}
\item \textbf{Current Courses:} Bayesian inference, classical inference, data analysis (R)

\item \textbf{Past Courses:} linear models (generalized inverses, Gauss-Markov/generalized models) R-programming and computing, probability (measure theory), real analysis (measure theory, basic functional analysis),
complex analysis, algebra (groups, rings, fields, modules, vector spaces), arithmetic dynamics,
mathematical cryptography, mathematical statistics, stochastic processes (discrete Markov processes, Poisson processes)
\end{itemize} 
\end{rSection}


%----------------------------------------------------------------------------------------
%	Research Interests
%----------------------------------------------------------------------------------------
\begin{rSection}{Research Interests and Current Projects}

{\bf Arithmetic Dynamics:}  
\begin{itemize}
\item A requirement of my masters degree consisted in creating some introductory notes to some open problems in the area of arithmetic dynamics. The focus is centered on introducing the notion of height of algebraic numbers,  potential theory and the interplay between the two. In particular, given some polynomial $\phi \in \Z[z]$ with $deg(\phi) \geq 2$, we can consider the Julia set $\MJ_{\phi}$ of $\phi$, the canonical height $\hat{h}_\phi$ associated with $\phi$ and the equilibrium measure $\mu$ of $\MJ_{\phi}$, that is, the measure that minimizes the energy functional $\int_{\MJ_{\phi}^2}-log|x-y|d\nu^2$ over all Borel probability measures $\nu$ with support in $\MJ_{\phi}$. Then any sequence $(z_n)_{n \in \N} \subset \overline{\Q}$ with $deg(z_n) \rightarrow \infty$ and $\hat{h}_{\phi}(z_n) \rightarrow 0$ as $n \rightarrow \infty$ has the conjugates of $z_n$ equidistributing around $\MJ_\phi$. There are open problems regarding the existence of a lower bound for $\hat{h}_{\phi}(z)$ for $z$ not preperiodic and in some sense bounded, but there is no answer for even simple cases like $\phi(z) = z^2+c$ with $c \in \Z$. I periodically update the notes. (\href{https://github.com/carsonaj/Math/blob/master/Arithmetic%20Dynamics/Arithmetic%20Dynamics%20Notes.pdf}{{\color{blue} creative component}})
\end{itemize}

{\bf Programming:}
\begin{itemize}
\item I am currently working on a C library and a corresponding wrapper in python using ctypes that implements number fields and finite fields and elliptic curves over these fields. After implementing these objects, I will implement the elliptic curve El-Gamal cryptosystem.  (\href{https://github.com/carsonaj/ccalc}{{\color{blue} Number Fields Code}})
\end{itemize}

{\bf Risk:}
\begin{itemize}
	\item I am currently writing up some notes on risk and different tools to assess it along with some case studies. Currently in a very early stage. (\href{https://github.com/carsonaj/Math/blob/master/Portfolio%20Theory/Portfolio%20Theory%20Notes.pdf}{{\color{blue} Risk}})
\end{itemize}


\end{rSection}

\newpage

%----------------------------------------------------------------------------------------
%	Work Experience
%----------------------------------------------------------------------------------------

\begin{rSection}{Work Experience} 
{\bf Graduate Teaching Assistant } at Texas A\&M University \hfill {\em August 2020 - Present}\\
\strut\hfill{20hrs}
\vspace{2mm}

{\bf Math Teacher} at Pensacola High School \hfill {\em August 2019 - May 2020}\\
\strut\hfill{40+hrs}\\
{\bf Courses Taught:} 
\begin{itemize}
\item Honors Algebra II
\item Honors Precalculus 
\item IB Statistics
\end{itemize} 
\vspace{2mm}

{\bf Graduate Teaching Assistant} at Oklahoma State University \hfill {\em August 2016 - May 2018} \\
\strut\hfill{20hrs}\\
{\bf Courses Taught:} 
\begin{itemize}
\item Trigonometry (instructor of record)
\item Business Calculus (recitation)
\end{itemize} 

\end{rSection}

%----------------------------------------------------------------------------------------
%	Volunteer Experience
%----------------------------------------------------------------------------------------
\begin{rSection}{Volunteer Experience}
{\bf Volunteer} with Love Without Boundaries Cambodia \hfill {\em May 2018 - September 2018}\\
\strut\hfill{40+hrs}\\
{\bf Responsibilities:} 
\begin{itemize}
\item Taught English to grades 11 and 12 in Tuol Prasat High School, 
\item Assisted the LWB staff in writing donor reports.
\end{itemize}
\end{rSection}

%----------------------------------------------------------------------------------------
%	Skills 
%----------------------------------------------------------------------------------------
\begin{rSection}{Skills}

{\bf Computer Languages}
\begin{itemize}
\item Python (intermediate)
\item C (intermediate)
\item R (intermediate)
\end{itemize}

{\bf Languages}
\begin{itemize}
\item English (native)
\item Spanish (fluent)
\item Portuguese (basic)
\end{itemize}

\end{rSection}

%----------------------------------------------------------------------------------------
%	Awards and Honors
%----------------------------------------------------------------------------------------
\begin{rSection}{Awards and Honors}
Hazel Bucy Endowment Fund (2017)

Member of Phi Beta Kappa Honor Society (2016)

Litchenburg Family Scholarship for Mathematics (2014)

Department of Mathematics General  Scholarship (2014)
\end{rSection}
\newpage
%----------------------------------------------------------------------------------------
%	References
%----------------------------------------------------------------------------------------
\begin{rSection}{References}
Paul Fili, Department of Mathematics, Oklahoma State University, paul.fili@okstate.edu

Alan Noell, Department of Mathematics, Oklahoma State University, noell@math.okstate.edu

Igor Pritsker, Department of Mathematics, Oklahoma State University, igor@math.okstate.edu
\end{rSection}



\end{document}