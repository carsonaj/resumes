\documentclass[12pt]{amsart}
\usepackage[margin=1in]{geometry} 
\usepackage{amsmath,amsthm,amssymb,amsfonts,setspace}
\usepackage{url}
\usepackage{hyperref}
\usepackage{xcolor}


\newtheorem{thm}{Theorem}[section]
\newtheorem{lem}[thm]{Lemma}
\newtheorem{prop}[thm]{Proposition}
\newtheorem{cor}[thm]{Corollary}
\newtheorem{conj}{Conjecture}
\newtheorem{defn}[thm]{Definition}
\newtheorem{note}[thm]{Note}
\newtheorem{ex}[thm]{Exercise}


\newcommand{\al}{\alpha}
\newcommand{\be}{\beta} 
\newcommand{\del}{\delta} 
\newcommand{\Del}{\Delta}
\newcommand{\lam}{\lambda}  
\newcommand{\Lam}{\Lambda} 
\newcommand{\ep}{\epsilon}
\newcommand{\sig}{\sigma} 
\newcommand{\om}{\omega}
\newcommand{\Om}{\Omega}
\newcommand{\C}{\mathbb{C}}
\newcommand{\N}{\mathbb{N}}
\newcommand{\E}{\mathbb{E}}
\newcommand{\Z}{\mathbb{Z}}
\newcommand{\R}{\mathbb{R}}
\newcommand{\Q}{\mathbb{Q}}
\renewcommand{\P}{\mathbb{P}}
\newcommand{\MA}{\mathcal{A}}
\newcommand{\MB}{\mathcal{B}}
\newcommand{\MF}{\mathcal{F}}
\newcommand{\MG}{\mathcal{G}}
\newcommand{\ML}{\mathcal{L}}
\newcommand{\MN}{\mathcal{N}}
\newcommand{\MS}{\mathcal{S}}
\newcommand{\MP}{\mathcal{P}}
\newcommand{\ME}{\mathcal{E}}
\newcommand{\MT}{\mathcal{T}}
\newcommand{\MM}{\mathcal{M}}

\newcommand{\RG}{[0,\infty]}
\newcommand{\Rg}{[0,\infty)}
\newcommand{\limfn}{\liminf \limits_{n \rightarrow \infty}}
\newcommand{\limpn}{\limsup \limits_{n \rightarrow \infty}}
\newcommand{\limn}{\lim \limits_{n \rightarrow \infty}}
\newcommand{\convt}[1]{\xrightarrow{\text{#1}}}
\newcommand{\conv}[1]{\xrightarrow{#1}} 



 
\newcommand{\n}{\noindent} 
 
\begin{document}


\textbf{\hspace{7cm}\large Resume \hspace{4cm} \large Carson James}\\ \phantom{1} \hspace{13cm} carsonjames@sbcglobal.net\\ \phantom{1} \hspace{13.9cm} (405) 315-5881


\n\textbf{\large Education} \vspace{2mm}

\n MS Mathematics, 2016-2018, Oklahoma State University, GPA: 4.0\\
\n BA Mathematics, 2013-2016, Oklahoma State University, GPA: 3.929 \vspace{.5mm}\\

 \n \textbf{\large Experience} \vspace{2mm}

\n \textbf{Research/Academic:}
I have research experience in the area of arithmetic dynamics (see thesis here: \textcolor{blue}{ \href{https://github.com/carsonaj/Math/blob/master/Arithmetic\%20Dynamics/Arithmetic\%20Dynamics\%20Notes.pdf}{link}}). My peronal interests are mostly in the area of probability and optimization. As such I am self-learning stochastic integration with respect to semimartingales and will eventually learn about stochastic control (see notes here: \textcolor{blue}{ \href{https://github.com/carsonaj/Math/blob/master/Stochastic\%20Analysis/Stochastic\%20Processes\%20-\%20James.pdf}{link}}). \vspace{2mm}


\n  \textbf{Applied:} I have completed and am currently working on several mathematics, statistics and data science projects. Some bigger projects include: a stock price analysis project involving a mix of graph theory, time series analysis, distribution fitting and montecarlo methods, a cookie-cutter neural network project involving deep learning on various data-sets via the automatic creation of neural networks with convenient options for architecture and shape and exploratory projects involving the implementation of various algorithms such as the simplex method for linear programs in standard form or Ukkonen's algorithm for constructing suffix trees in linear time which are useful in other fields like operations research and bioinformatics respectively. \vspace{2mm}

\n \textbf{Programming:} All the applied mathematics projects that I work on are implemented in Python, C or both. Most of the higher level programming (for instance time series analysis) is done in Python since the object oriented capabilities are convenient. Lower level projects like implementing the simplex method or Ukkonen's algorithm are implemented in both Python and C to observe the difference in performance. Most of the libraries that I have made are available here: \textcolor{blue}{ \href{https://github.com/carsonaj/Programming}{link}} \vspace{2mm}

\n \textbf{Teaching:}\\ 
- four sections of business calculus (recitation instructor)\\
- two sections of trigonometry (instructor)\\
- two Cambodian highschool English classes, grades 11,12 (instructor)\\

\smallskip \n \textbf{\large Skills}\vspace{2mm}\\
\n \textbf{Software:}\\
-Python: numpy, pandas, matplotlib, scipy, statsmodels, pytorch\\
-C\\
-Linux (Basic)\vspace{2mm}

\n \textbf{Languages:}\\
English (native)\\
Spanish (fluent)

\end{document}